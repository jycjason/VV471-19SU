\documentclass[a4paper,12pt,titlepage]{article}
\usepackage{amsmath} 
\usepackage{amssymb}
\usepackage[nottoc]{tocbibind}
\usepackage{float}
\usepackage{indentfirst}
\author{\textit{Jiang Yicheng}\\\textit{515370910224}}
\title{\textbf{VE203\\
		Assignment 2}}
\date{\today}

\usepackage[top=1 in, bottom=0.8 in, left= 1in, right=1 in]{geometry}
\usepackage{fancyhdr,lastpage}
	\pagestyle{fancy}
	\fancyhf{}
\cfoot{Page \thepage\ of \pageref{LastPage}}
\usepackage{multirow}
\usepackage{gauss}

\begin{document}

\maketitle

\section{2 + 2 = 4}
\subsection{}
We define the function $\cdot+\cdot:\mathbb{N}\times\mathbb{N}\mapsto \mathbb{N}$ has the following properties:
\begin{enumerate}
\item $n+0=n$ 
\item For $m\neq0$, set $m=succ(m')$, $n+m=succ(n)+m'$
\end{enumerate}

\subsection{}
Since $n+1=succ(n),2=succ(1),3=succ(2),4=succ(3)$, then
\begin{align*}
2+2=2+succ(1)=succ(2)+1=3+1=succ(3)=4
\end{align*}

So we get that $2+2=4$

\subsection{}
We first prove that $\forall m,n\in\mathbb{N}, succ(n+m)=succ(n)+m$

$\forall n\in\mathbb{N}$,
\begin{enumerate}
\item When $m=0$, $succ(n+m)=succ(n+0)=succ(n)=succ(n)+0=succ(n)+m$
so the statement holds when $m=0$
\item Assume that the statement holds when $m=m_0$, i.e. $succ(n+m_0)=succ(n)+m_0$, then according to definition of addition
\begin{align*}
succ(n+succ(m_0))&=succ(succ(n)+m_0)\\
&=succ(succ(n))+m_0\\
&=succ(n)+succ(m_0)
\end{align*}
\end{enumerate}

so the statement also holds when $m=succ(m_0)$.

To sum up, $\forall m,n\in\mathbb{N}, succ(n+m)=succ(n)+m$. Then we try to prove that  $\forall n\in\mathbb{N}, n+0=0+n$ 
\begin{enumerate}
\item When $n=0$, $n+0=0+0=0+n$
so the statement holds when $n=0$
\item Assume that the statement holds when $n=n_0$, i.e. $n_0+0=0+n_0$, then according to definition of addition and the statement we just prove
\begin{align*}
0+succ(n_0)&=succ(0)+n_0=succ(0+n_0)\\
&=succ(n_0+0)=succ(n_0)\\
&=succ(n_0)+0
\end{align*}
\end{enumerate}

so the statement also holds when $n=succ(n_0)$.

To sum up, $\forall n\in\mathbb{N}, n+0=0+n$. 

Now we start to prove that  $\forall m,n\in\mathbb{N}, n+m=m+n$.  $\forall n\in \mathbb{N}$ 
\begin{enumerate}
\item When $m=0$, $n+0=0+n$ has been proved , so the statement holds when $m=0$
\item Assume that the statement holds when $m=m_0$, i.e. $n+m_0=m_0+n$, then according to definition of addition and the statement we just prove
\begin{align*}
n+succ(m_0)&=succ(n)+m_0=succ(n+m_0)\\
&=succ(m_0+n)\\
&=succ(m_0)+n
\end{align*}
\end{enumerate}

so the statement also holds when $m=succ(m_0)$.

In conclusion, $\forall m,n\in\mathbb{N}, n+m=m+n$.

\section{Straightforward Induction}
\paragraph{Proof:}
\begin{enumerate}
\item When $n=1,2$, since $a_1=1=3\cdot2^{1-1}+2\cdot(-1)^1$, $a_2=8=3\cdot2^{2-1}+2\cdot(-1)^2$, so $a_n=3\cdot2^{n-1}+2\cdot(-1)^n$ holds for $n=1,2$

\item Assume that $\forall n\leqslant k(k\geqslant2)$, $a_n=3\cdot2^{n-1}+2\cdot(-1)^n$ holds, then
\begin{align*}
a_{k+1}&=a_k+2a_{k-1}=3\cdot2^{k-1}+2\cdot(-1)^k+2\cdot(3\cdot2^{k-1-1}+2\cdot(-1)^{k-1})\\
&=(3+3)\cdot2^{k-1}+(2-4)\cdot(-1)^k\\
&=3\cdot2^{k+1-1}+2\cdot(-1)^{k+1}
\end{align*}
\end{enumerate}

so the equation also holds for $n=k+1$.

To sum up, $\forall n>0,a_n=3\cdot2^{n-1}+2\cdot(-1)^n$


\section{The Fifth Peano Axiom}
\paragraph{Proof:}Assume that there exists some non-empty set $S \subset \mathbb{N}$ doesn't have a least element, that is $\forall m\in S, \exists m_0\in S$ such that $m_0<m$.

Set $T=\mathbb{N}\setminus S$. Since 0 is not the successor of any natural number, then if $0\in S$, it would be the least number in $S$. So $0\in T$.

Since 1 is the successor of $0$, and since $0\notin S$, then if $1\in S$, 1 would be the least number in $S$. So $1\in T$. 

Since 2 is the successor of $1$, and since $0,1\notin S$, then if $2\in S$, 2 would be the least number in $S$. So $2\in T$.

Repeat this way, we can prove that for each natural number $n$, its successor is in $T$. And since $0\in T$, then according to induction axiom $T=\mathbb{N}$, and therefore $S=\varnothing$. This leads to contradiction.

So such kind of set doesn't exist, and \textit{Well-Ordering Principle} holds.


\section{Is a direct induction approach always successful?}
First, we use induction to prove that  $(1+x)^n\geqslant1+nx$ holds for any n $\in \mathbb{N}$, where $x > -1$
\paragraph{Proof:}
\begin{enumerate}
	\item When n = 0, $(1+x)^n = (1+x)^0 = 1,1+nx = 1+0 \cdot x=1$. So $(1+x)^n \geqslant 1+nx$ holds when n=0
	\item Assume that $(1+x)^n \geqslant 1+nx$ holds when n=k, where k $\in \mathbb{N}$, i.e. $(1+x)^k \geqslant 1+kx$. Then 
\begin{align*}
(1+x)^{k+1} &\geqslant (1+x) \cdot (1+kx)  \\
	&=1+kx+x+kx^2\\
	&=(1+(k+1)x+kx^2\\
	&\geqslant 1+(k+1)x 
\end{align*}	
This is because $k \in \mathbb{N}, x>-1$	

So $(1+x)^n\geqslant1+nx$ also holds when $n=k+1$. 
\end{enumerate}

According to $1,2, \forall n \in\mathbb{N}$, $(1+x)^n\geqslant1+nx$.

So $ \forall n \in \mathbb{N}, (1+x)^n \geqslant nx$

\section{Strong Induction}
\paragraph{Proof:}
\begin{enumerate}
\item When $n=1,2$, since $1=2^0,2=2^1$, then we can see that the statement holds for $n=1,2$.
\item Assume that $\forall n\in\mathbb{N},1\leqslant n\leqslant k(k\geqslant2)$, it can be written as a sum of distinct powers of 2. Then for $n=k+1$:
\begin{enumerate}
\item If $k+1$ is even, then $k\geqslant\dfrac{k+1}{2}\in\mathbb{N}^*$, according to assumption we can set that $\dfrac{k+1}{2}=p_02^{0}+p_12^1+\cdots+p_{(k+1)/2}2^{(k+1)/2}(p_i\in\lbrace0,1\rbrace,i=0,1,\cdots,(k+1)/2)$. This is practical since $2^n=(1+1)^n>n\cdot1=n$. Then $k+1=p_02^{1}+p_12^2+\cdots+p_{(k+1)/2}2^{(k+1)/2+1}$ which is a sum of distinct powers of 2.

\item If $k+1$ is odd, then $\dfrac{k}{2}\in\mathbb{N}^*,1\leqslant\dfrac{k}{2}\leqslant k$, according to assumption we can set that $\dfrac{k}{2}=p_02^{0}+p_12^1+\cdots+p_{k/2}2^{k/2}(p_i\in\lbrace0,1\rbrace,i=0,1,\cdots,k/2)$. Then $k+1=2^0+p_02^{1}+p_12^2+\cdots+p_{k/2}2^{k/2+1}$ which is a sum of distinct powers of 2.
\end{enumerate}
according to (a)(b), $k+1$ always can be written as a sum of distinct powers of 2. So, the statement also holds for $n=k+1$.
\end{enumerate}

From 1,2, every $n\in\mathbb{N}\setminus\lbrace0\rbrace$ can be written as a sum of distinct powers of 2.

\section{Structural Induction}
\paragraph{Proof:}Use structural induction to prove
\begin{enumerate}
\item Since $(0,0)\in S$, and $5\big|0+0$, the base case is established.

\item Assume that $\forall a,b\in S, 5\big|(a+b)$, so we can set $a+b=5k,k\in\mathbb{N}$. Then $(a+2,b+3),(a+3,b+2)\in S$, and $(a+2)+(b+3)=(a+3)+(b+2)=a+b+5=5(k+1)$. Since $k+1\in \mathbb{N}$, $5\big| ((a+2)+(b+3)),5\big|((a+3)+(b+2))$.


\end{enumerate}

According to 1,2, we can see that $\forall (a,b)\in S, 5\big|(a+b)$

\section{Some easy practice of relation properties}
\subsection{x+y=0}
\begin{enumerate}
\item Since $1\in\mathbb{Z},1+1=2\neq0$, this shows that the relation is not reflexive.
\item $\forall (x,y)\in R, y+x=x+y=0$, so $(y,x)\in R$. So the relation is symmetric.
\item Since $1+(-1)=0,(-1)+1=0$, then $(1,-1),(-1,1)\in R$, while $1+1=2\neq0$,  so the relation is not transitivity.
\end{enumerate}

\subsection{2$\big|$(x-y)}
\begin{enumerate}
\item $\forall x\in\mathbb{Z}, (x,x)\in R$ since $2\big|0=x-x$, which shows that the relation is reflexive.
\item $\forall (x,y)\in R$, set $x-y=2k,k\in\mathbb{Z}$. Then $y-x=-2k=2(-k)$. Since $-k\in\mathbb{Z}, 2\big|(y-x)$. So $(y,x)\in R$. So the relation is symmetric.
\item $\forall (x,y),(y,z)\in R$, set $ x-y=2k_1, y-z=2k_2(k_1,k_2\in\mathbb{Z})$. Then $x-z=x-y+y-z=2(k_1+k_2)$. Since $k_1+k_2\in\mathbb{Z}$, $2\big|(x-z)$. So $(x,z)\in R$. So the relation is transitivity.
\end{enumerate}


\subsection{xy=0}
\begin{enumerate}
\item Since $1\in\mathbb{Z},1\cdot1=1\neq0$, this shows that the relation is not reflexive.
\item $\forall (x,y)\in R, yx=xy=0$, so $(y,x)\in R$. So the relation is symmetric.
\item Since $1\cdot 0=0,0\cdot 2=0$, then $(1,0),(0,2)\in R$, while $1\cdot2=2\neq0$,  so the relation is not transitivity.
\end{enumerate}

\subsection{x=1 or y=1}
\begin{enumerate}
\item Since $2\in\mathbb{Z},2\neq1$, then $(2,2)\notin R$. This shows that the relation is not reflexive.
\item $\forall (x,y)\in R, x=1\vee y=1$ , so $(y,x)\in R$. So the relation is symmetric.
\item Since $(3,1),(1,2)\in R$, while $(3,2)\notin R$, then the relation is not transitivity.
\end{enumerate}

\subsection{x=$\pm$ y}
\begin{enumerate}
\item $\forall x\in\mathbb{Z}, (x,x)\in R$ since $x=x$, which shows that the relation is reflexive.
\item $\forall (x,y)\in R, x=y\vee x=-y$. So $y=x\vee y=-x$. So $(y,x)\in R$. So the relation is symmetric.
\item $\forall (x,y),(y,z)\in R x=y\vee x=-y, y=z\vee y=-z$. Then $x=z\vee x=-z$. So $(x,z)\in R$. So the relation is transitivity.
\end{enumerate}




\subsection{x=2y}
\begin{enumerate}
\item Since $1\in\mathbb{Z},1\neq2=2\cdot1$, then $(1,1)\notin R$. This shows that the relation is not reflexive.
\item $(2,1)\in R$, while $(1,2)\notin R$. So the relation is not symmetric.
\item Since $(4,2),(2,1)\in R$, while $(4,1)\notin R$, then the relation is not transitivity.
\end{enumerate}


\subsection{xy$\geqslant0$}
\begin{enumerate}
\item $\forall x\in\mathbb{Z}, (x,x)\in R$ since $x\cdot x=x^2\geqslant0$, which shows that the relation is reflexive.
\item $\forall (x,y)\in R, yx=xy\geqslant0$. So $(y,x)\in R$. So the relation is symmetric.
\item Since $1\cdot 0=0\geqslant0,0\cdot (-1)=0\geqslant0$, then $(1,0),(0,-1)\in R$, while $1\cdot(-1)=-1<0$,  so the relation is not transitivity.
\end{enumerate}



\subsection{x=1}
\begin{enumerate}
\item Since $2\in\mathbb{Z},2\neq1$, then $(2,2)\notin R$. This shows that the relation is not reflexive.
\item $(1,2)\in R$, while $(2,1)\notin R$, so the relation is not symmetric.
\item $\forall x,y,z\in\mathbb{Z},(x,y),(y,z)\in R$, then $x=1$ and therefore $(x,z)\in R$, then the relation is transitivity.
\end{enumerate}

\end{document}
