\documentclass[a4paper,12pt,titlepage]{article}
\usepackage{amsmath} 
\usepackage{amssymb}
\usepackage[nottoc]{tocbibind}
\usepackage{mathrsfs}
\usepackage{float}
\usepackage{indentfirst}
\author{\textit{Jiang Yicheng}\\\textit{515370910224}}
\title{\textbf{VE203\\
		Assignment 4}}
\date{\today}
\usepackage{dsfont}
\usepackage[top=0.8in, bottom=0.8in, left= 2cm, right=2cm]{geometry}
\usepackage{fancyhdr,lastpage}
	\pagestyle{fancy}
	\fancyhf{}
\cfoot{Page \thepage\ of \pageref{LastPage}}
\usepackage{multirow}
\usepackage{gauss}

\begin{document}

\maketitle

\section*{Exercise 4.1} 
\subsection*{i)}
$$247=13\times 19,1=3\times13-2\times19$$
So the two primes are $p_1=13,p_2=19$, and $x=3,y=-2$ such that $1=p_1\cdot x+p_2\cdot y$.
\subsection*{ii)}
$$10^{100}\equiv (-3)^{100}\equiv 3^{100}\equiv(27)^{33}\cdot3\equiv1^{33}\cdot3\equiv3 \,\,(mod\,\,\,13)$$

$$10^{100}\equiv100^{50}\equiv5^{50}\equiv25^{25}\equiv6^{25}\equiv(-2)^{12}\cdot6\equiv16^{3}\cdot6\equiv(-3)^3\cdot6\equiv11\cdot6\equiv9\,\,(mod\,\,\,19)$$

So $10^{100}\equiv3 \,\,(mod\,\,\,13),10^{100}\equiv9\,\,(mod\,\,\,19)$

\subsection*{iii)}
Because $11\cdot19\equiv-2\cdot6\equiv1\,\,(mod \,\,\,13)$, $3\cdot13\equiv1\,\,(mod\,\,\,19)$, then according to Chinese Remainder Theorem
$$10^{100}\equiv3\cdot 11\cdot 19+9\cdot 3\cdot 13\equiv978\equiv237\,\,(mod\,\,\,247)$$
so $r=237.$

\section*{Exercise 4.2}

We see that $2^8=256=2\cdot99+58$, so
$$4^8\equiv4^2\equiv7\,\,(mod\,\,\,9),2^8\equiv3\,\,(mod\,\,\,11)$$
So $n=8$ satisfies both $4^n\equiv7\,\,(mod\,\,\,9),2^n\equiv3\,\,(mod\,\,\,11)$  


\section*{Exercise 4.3}
$$45029^2<2027651281<45030^2$$
We need to calculate $k^2-2027651281$ for
$$45029<k<\dfrac{2027651281+1}{2}=1013825641$$
We find:
$$45030^2-2027651281=49619,
45031^2-2027651281=139680$$ 

$$45032^2-2027651281=229743,
45033^2-2027651281=319808$$ 

$$45034^2-2027651281=409875,
45035^2-2027651281=499944$$

$$45036^2-2027651281=590015,
45037^2-2027651281=680088$$

$$45038^2-2027651281=770163,
45039^2-2027651281=860240$$

$$45040^2-2027651281=950319,
45041^2-2027651281=1040400=1020^2$$   

So $2027651281=45041^2-1020^2=46061\cdot44021$. And we can check that both $46061$ and 44021 are primes. So the factors of 2027651281 are $1,44021,46061,2027651281.$
 
 
 \section*{Exercise 4.4}
According to Fermat's Little Theorem, 
$$5^{7-1}\equiv1\,\,(mod\,\,\,7),5^{11-1}\equiv1\,\,(mod\,\,\,11),5^{13-1}\equiv1\,\,(mod\,\,\,13)$$ 
so
$$5^{2003}\equiv5^{333\cdot6+5}\equiv1^{333}\cdot4^2\cdot5\equiv2\cdot5\equiv3\,\,(mod\,\,\,7)$$ 
$$5^{2003}\equiv5^{200\cdot10+3}\equiv1^{200}\cdot125\equiv4\,\,(mod\,\,\,11)$$  
$$5^{2003}\equiv5^{166\cdot12+11}\equiv1^{166}\cdot(-1)^5\cdot5\equiv8\,\,(mod\,\,\,13)$$  
 
Since $5\cdot11\cdot13\equiv(-2)\cdot4\cdot(-1)\equiv1\,\,(mod\,\,\,7)$, $4\cdot7\cdot13\equiv4\cdot(-4)\cdot2\equiv1\,\,(mod\,\,\,11)$,
$12\cdot7\cdot11\equiv(-1)\cdot7\cdot(-2)\equiv1\,\,(mod\,\,\,13)$, then according to Chinese Remainder Theorem,
$$5^{2003}\equiv3\cdot5\cdot11\cdot13+
4\cdot4\cdot7\cdot13+8\cdot12\cdot7\cdot11
\equiv10993\equiv983\,\,(mod\,\,\,1001)$$   

So $5^{2003}\equiv 983\,\,(mod\,\,\,1001)$. 
\section*{Exercise 4.5} 
\subsection*{i)} 
Assume that $(p-1)!\equiv-1\,\,(mod\,\,\,p)$ while $p$ is not a prime. Set $p=a\cdot b, a,b\in\mathbb{N},a\leqslant b$.
 
Then $a,b\in\mathbb{N}\cap[1,p-1]$, so $c:=\dfrac{(p-1)!}{a}\in\mathbb{N}$ and 
$$b\cdot(p-1)!\equiv b\cdot a\cdot \dfrac{(p-1)!}{a}\equiv p\cdot c\equiv0\,\,(mod\,\,\,p)$$

While $(p-1)!\equiv-1\,\,(mod\,\,\,p)$, then
$$b\cdot (p-1)!\equiv -b \,\,(mod\,\,\,p)$$
So $-b\equiv0\,\,(mod\,\,\,p)$. Since $b\in\mathbb{N}\cap[1,p-1]$, this is impossible. So our assumption is wrong.

So $p$ is a prime.
\subsection*{ii)}
$\forall a\in\mathbb{N}\cap[1,m-1], a\equiv -(m-a)\,\,(mod\,\,\,m)$, so for any odd integer $m,z=\dfrac{m-1}{2}$,
\begin{align*}
(m-1)!\equiv 
\prod_{i=1}^zi\cdot 
\prod_{i=z+1}^{m-1}i\equiv\prod_{i=1}^zi\cdot 
\prod_{i=z+1}^{m-1}-(m-i)\equiv\prod_{i=1}^zi\cdot 
(-1)^z\prod_{j=1}^{z}j\equiv (-1)^z(z!)^2\,\,(mod\,\,\,m)
\end{align*}

So for any odd integer $m$, $(m-1)!\equiv 
(-1)^z(z!)^2\,\,(mod\,\,\,m)$

\subsection*{iii)}
To see whether an odd integer $m$ is a prime, we can check whether $(-1)^z(z!)^2\equiv-1\,\,(mod\,\,\,m)$ where $z=\dfrac{m-1}{2}$.

From $i)ii)$, the method is correct. Then we need to see whether the method is practical.
It seems that we haven't an easy way to calculate $ z!\,\,\,mod\,\,\,m$ and therefore the method will lead to a complex calculation. However, this is a new way which can be implemented by computer. With proper programme, it can work well.   

\section*{Exercise 4.6}
\subsection*{i)}
$$x\equiv 0\,\,(mod\,\,\,11)\Rightarrow x^2\equiv 0\,\,(mod\,\,\,11),x\equiv 1\,\,(mod\,\,\,11)\Rightarrow x^2\equiv 1\,\,(mod\,\,\,11)$$
$$x\equiv 2\,\,(mod\,\,\,11)\Rightarrow x^2\equiv 4\,\,(mod\,\,\,11),x\equiv 3\,\,(mod\,\,\,11)\Rightarrow x^2\equiv 9\,\,(mod\,\,\,11)$$
$$x\equiv 4\,\,(mod\,\,\,11)\Rightarrow x^2\equiv 5\,\,(mod\,\,\,11),x\equiv 5\,\,(mod\,\,\,11)\Rightarrow x^2\equiv 4\,\,(mod\,\,\,11)$$
$$x\equiv 6\,\,(mod\,\,\,11)\Rightarrow x^2\equiv 3\,\,(mod\,\,\,11),x\equiv 7\,\,(mod\,\,\,11)\Rightarrow x^2\equiv 5\,\,(mod\,\,\,11)$$
$$x\equiv 8\,\,(mod\,\,\,11)\Rightarrow x^2\equiv 9\,\,(mod\,\,\,11),x\equiv 9\,\,(mod\,\,\,11)\Rightarrow x^2\equiv 4\,\,(mod\,\,\,11)$$
$$x\equiv 10\,\,(mod\,\,\,11)\Rightarrow x^2\equiv 1\,\,(mod\,\,\,11)$$

So $x^2\equiv a\,\,(mod\,\,\,11)$ has a solution if and only if $a\equiv0,1,3,4,5,9\,\,(mod\,\,\,11)$. Taking $gcd(a,11)=1$ into account,  $1+11t,3+11t,4+11t,5+11t,9+11t,t\in\mathbb{Z}$ are quadratic residues of 11.


\subsection*{ii)}
For any $a\in\mathbb{Z},p\nmid a$, then 
\begin{enumerate}
\item $x^2\equiv a\,\,(mod\,\,\,p)$ has no solutions modulo $p$
\item $x^2\equiv a\,\,(mod\,\,\,p)$ has a solution modulo $p$: $x\equiv b\,\,(mod\,\,\,p),b\in\mathbb{N}, $ then for some $x$ such that $x\equiv p-b\,\,(mod\,\,\,p)$, we can see $x^2\equiv (p-b)^2\equiv b^2\equiv a\,\,(mod\,\,\,p)$. If $p-b\equiv b\,\,(mod\,\,\,p)$, then $2b\equiv p\equiv 0\,\,(mod\,\,\,p)$. Since $p$ is an odd prime, $b\equiv 0\,\,(mod\,\,\,p)$. So $a\equiv b^2\equiv 0\,\,(mod\,\,\,p)$ which leads to contradiction. So $x\equiv p-b\,\,(mod\,\,\,p)$ and   $x\equiv b\,\,(mod\,\,\,p)$ are two incongruent solutions modulo $p$.

If $x\equiv c\,\,(mod\,\,\,p)$ is another solution to $x^2\equiv a\,\,(mod\,\,\,m)$ where $c\in \mathbb{N}, c\not \equiv b\,\,(mod\,\,\,p)\wedge c\not \equiv p-b\,\,(mod\,\,\,p)$, then $c^2\equiv a\equiv b^2\,\,(mod\,\,\,p)$, furthermore, $p|(c-b)(c+b)$. Since $p$ is a prime, then $p|(c-b)$ or $p|(c+b)$. Since $c\not \equiv b\,\,(mod\,\,\,p)\wedge c\not \equiv p-b\,\,(mod\,\,\,p)$, this is contradiction. 

So if $x^2\equiv a\,\,(mod\,\,\,p)$ has solutions, it exactly has two incongruent solutions modulo $p$.
\end{enumerate}

To sum up, the congruence $x^2\equiv a\,\,(mod\,\,\,p)$ has either no solutions or exactly two incongruent solutions modulo $p$.

\subsection*{iii)}
From $ii)$ we know that $\forall b\in\mathbb{Z}\cap[1,\dfrac{p-1}{2}]$, $x\equiv b\,\,(mod\,\,\,p)$ and $x\equiv p-b\,\,(mod\,\,\,p)$ both lead to $x^2\equiv b^2\,\,(mod\,\,\,p)$, and for different $b$, $b^2$ are incongruent modulo $p$. So for any odd prime, $\forall x\in\mathbb{Z}$, $x^2$ has  $\dfrac{p-1}{2}$ different results modulo $p$ (except 0). And therefore for exactly $\dfrac{p-1}{2}$ incongruent numbers $a$ among $1,2,\cdots,p-1$, $x^2\equiv a\,\,(mod\,\,\,p)$ has solution.(These are all possible result for $x^2\,\,mod\,\,p$ except 0, so all quadratic residues modulo $p$ are among them.) Moreover for any number $a$ among these numbers, $gcd(a,p)=1$.

So if $p$ is an odd prime, then there are exactly $\dfrac{p-1}{2}$ quadratic residues of $p$ among the integers $1, 2,\cdots,p - 1$.

\subsection*{iv)}
Since $a\equiv b\,\,(mod\,\,\,p)$, then
$x^2\equiv b\equiv a\,\,(mod\,\,\,p)$. So if $x^2\equiv a\,\,(mod\,\,\,p)$ has solution and $gcd(a,p)=1$, then $x^2\equiv b\,\,(mod\,\,\,p)$ must have a solution and $gcd(b,p)=gcd(a+kp,p)=gcd(a,p)=1$ where $k$ is an integer; if $x^2\equiv a\,\,(mod\,\,\,p)$ doesn't have a solution, neither will $x^2\equiv b\,\,(mod\,\,\,p)$; if $gcd(a,p)\neq 1$, $gcd(b,p)=gcd(a,p)\neq 1$.

So if $a$ is a quadratic residue, so will $b$; and if $a$ isn't a quadratic residue, neither will $b$. So 
$$\Big(\dfrac{a}{p}\Big)=\Big(\dfrac{b}{p}\Big)$$

\subsection*{v)}
If $a$ is a quadratic residue of $p$, then $\exists x\in\mathbb{Z}\cap[1,p-1]$ such that $x^2\equiv a\,\,(mod\,\,\,p)$. Since $p$ is a prime and $p\nmid x$, according to Fermat's Little Theorem,
$$x^{p-1}\equiv 1\,\,(mod\,\,\,p)$$
Since $\Big(\dfrac{a}{p}\Big)=1$ when $a$ is a quadratic residue of $p$,  
$$a^{(p-1)/2}\equiv(x^2)^{(p-1)/2}\equiv x^{p-1}\equiv 1\equiv \Big(\dfrac{a}{p}\Big)\,\,(mod\,\,\,p)$$

If  $a$ is not a quadratic residue of $p$, 

$\forall m\in\mathbb{N}\cap[1,p-1]$,  $gcd(m,p)=1$, so there exists a unique $ n_0\in\mathbb{N}\cap[1,p-1]$ such that
$m\cdot n_0\equiv1\,\,(mod\,\,\,p)$, and therefore 
$$m\cdot(a\cdot n_0)\equiv a\,\,(mod\,\,\,p)$$
If $a\cdot n_0\equiv a\cdot n_0'\,\,(mod\,\,\,p)$, then since $a$ is not divisible by $p$ and $p$ is a prime, $gcd(a,p)=1$ and $n_0\equiv n_0'\,\,(mod\,\,\,p)$. So $\forall m\in\mathbb{N}\cap[1,p-1]$, there exists a unique $ n\in\mathbb{N}\cap[1,p-1]$ such that
$m\cdot n\equiv a\,\,(mod\,\,\,p)$, and for different $m$, $n$ will be different. Since $a$ is not a quadratic residue of $p$, $m\neq n$. So for $1,2,\cdots,p-1$, they can be
grouped into $\dfrac{p-1}{2}$ pairs $m, n$ where $m\neq n$ and $m\cdot n\equiv a\,\,(mod\,\,\,p)$, and therefore
$$2\cdot 3\cdot\cdots(p-1)\equiv a^{(p-1)/2}\,\,(mod\,\,\,p)$$
According to Wilson's Theorem, 
$$a^{(p-1)/2}\equiv(p-1)!\equiv -1\equiv \Big(\dfrac{a}{p}\Big)\,\,(mod\,\,\,p)$$

To sum up, if $p$ is an odd prime and $a$ is a positive integer not divisible by $p$, then
$$a^{(p-1)/2}\equiv \Big(\dfrac{a}{p}\Big)\,\,(mod\,\,\,p)$$

\subsection*{vi)}
According to $v)$, we obtain that if $p$ is an odd prime and $a$ and $b$ are integers not divisible by $p$,
 $$\Big(\dfrac{ab}{p}\Big)\equiv(ab)^{(p-1)/2}\equiv a^{(p-1)/2}\cdot b^{(p-1)/2}\equiv \Big(\dfrac{a}{p}\Big)\Big(\dfrac{b}{p}\Big)\,\,(mod\,\,\,p)$$

Since $\Big(\dfrac{a}{p}\Big),\Big(\dfrac{b}{p}\Big),\Big(\dfrac{ab}{p}\Big)\in\lbrace1,-1\rbrace$, $\Big(\dfrac{ab}{p}\Big)= \Big(\dfrac{a}{p}\Big)\Big(\dfrac{b}{p}\Big)$.

\subsection*{vii)}
If $p$ is an odd prime and $a$ is a positive integer not divisible by $p$, then according to Fermat's Little Theorem,
$$a^{p-1}\equiv1\,\,(mod\,\,\,p)$$
so $(a^{(p-1)/2}+1)(a^{(p-1)/2}-1)\equiv0\,\,(mod\,\,\,p)$. Since $p$ is a prime,
$$a^{(p-1)/2}\equiv-1\,\,(mod\,\,\,p)\vee a^{(p-1)/2}\equiv1\,\,(mod\,\,\,p)$$
then with $v)$ we obtain that: If $p$ is an odd prime and $a$ is a positive integer not divisible by $p$, then
\begin{enumerate}
\item $a$ is a quadratic residue of $p$ if and only if $a^{(p-1)/2}\equiv1\,\,(mod\,\,\,p)$
\item $a$ is not a quadratic residue of $p$ if and only if $a^{(p-1)/2}\equiv-1\,\,(mod\,\,\,p)$
\end{enumerate}

If $-1$ is a quadratic residue of $p$ ($p\nmid-1$), then
$$(-1)^{(p-1)/2}\equiv 1\,\,(mod\,\,\,p)$$
so $(p-1)/2=2k$ where $k\in\mathbb{Z}$. So $p=4k+1$ which implies that 
$$p\equiv1\,\,(mod\,\,\,4)$$ 

If $-1$ is not a quadratic residue of $p$ ($p\nmid-1$), then
$$(-1)^{(p-1)/2}\equiv -1\,\,(mod\,\,\,p)$$
so $(p-1)/2=2k+1$ where $k\in\mathbb{Z}$. So $p=4k+3$ which implies that 
$$p\equiv3\,\,(mod\,\,\,4)$$ 

To sum up, if $p$ is an odd prime, then  $-1$ is a quadratic residue of $p$ if $p\equiv1\,\,(mod\,\,\,4)$, and $-1$ is not a quadratic residue of $p$ if $p\equiv3\,\,(mod\,\,\,4)$.

\subsection*{viii)}
$x^2\equiv29\,\,(mod\,\,\,35)\Rightarrow x^2\equiv29\equiv4\,\,(mod\,\,\,5)\wedge x^2\equiv29\equiv1\,\,(mod\,\,\,7)$

$$x\equiv 0\,\,(mod\,\,\,5)\Rightarrow x^2\equiv 0\,\,(mod\,\,\,5),x\equiv 1\,\,(mod\,\,\,5)\Rightarrow x^2\equiv 1\,\,(mod\,\,\,5)$$
$$x\equiv 2\,\,(mod\,\,\,5)\Rightarrow x^2\equiv 4\,\,(mod\,\,\,5),x\equiv 3\,\,(mod\,\,\,5)\Rightarrow x^2\equiv 4\,\,(mod\,\,\,5)$$
$$x\equiv 4\,\,(mod\,\,\,5)\Rightarrow x^2\equiv 1\,\,(mod\,\,\,5)$$
So $x^2\equiv4\,\,(mod\,\,\,5)\Leftrightarrow x\equiv 2\,\,(mod\,\,\,5)\vee x\equiv 3\,\,(mod\,\,\,5)$.

$$x\equiv 0\,\,(mod\,\,\,7)\Rightarrow x^2\equiv 0\,\,(mod\,\,\,7),x\equiv 1\,\,(mod\,\,\,7)\Rightarrow x^2\equiv 1\,\,(mod\,\,\,7)$$
$$x\equiv 2\,\,(mod\,\,\,7)\Rightarrow x^2\equiv 4\,\,(mod\,\,\,7),x\equiv 3\,\,(mod\,\,\,7)\Rightarrow x^2\equiv 2\,\,(mod\,\,\,7)$$
$$x\equiv 4\,\,(mod\,\,\,7)\Rightarrow x^2\equiv 2\,\,(mod\,\,\,7),x\equiv 5\,\,(mod\,\,\,7)\Rightarrow x^2\equiv 4\,\,(mod\,\,\,7)$$
$$x\equiv 6\,\,(mod\,\,\,7)\Rightarrow x^2\equiv 1\,\,(mod\,\,\,7)$$
So $x^2\equiv1\,\,(mod\,\,\,7)\Leftrightarrow x\equiv 1\,\,(mod\,\,\,7)\vee x\equiv 6\,\,(mod\,\,\,7)$.

Because $3\cdot7\equiv1\,\,(mod \,\,\,5)$, $3\cdot5\equiv1\,\,(mod\,\,\,7)$, then according to Chinese Remainder Theorem
\begin{enumerate}

\item $x\equiv 2\,\,(mod\,\,\,5)\wedge x\equiv 1\,\,(mod\,\,\,7)$
$$x\equiv2\cdot 3\cdot 7+1\cdot 3\cdot 5\equiv57\equiv22\,\,(mod\,\,\,35)$$
so $x=22+35t,t\in\mathbb{Z}.$

\item $x\equiv 2\,\,(mod\,\,\,5)\wedge x\equiv 6\,\,(mod\,\,\,7)$
$$x\equiv2\cdot 3\cdot 7+6\cdot 3\cdot 5\equiv132\equiv27\,\,(mod\,\,\,35)$$
so $x=27+35t,t\in\mathbb{Z}.$

\item $x\equiv 3\,\,(mod\,\,\,5)\wedge x\equiv 1\,\,(mod\,\,\,7)$
$$x\equiv3\cdot 3\cdot 7+1\cdot 3\cdot 5\equiv78\equiv8\,\,(mod\,\,\,35)$$
so $x=8+35t,t\in\mathbb{Z}.$

\item $x\equiv 3\,\,(mod\,\,\,5)\wedge x\equiv 6\,\,(mod\,\,\,7)$
$$x\equiv3\cdot 3\cdot 7+6\cdot 3\cdot 5\equiv153\equiv13\,\,(mod\,\,\,35)$$
so $x=13+35t,t\in\mathbb{Z}.$




\end{enumerate}
With simple check we can see that all these are solutions to the origin congruence.

So the solution set of the congruence $x^2 = 29 \,\,(mod\,\,\,35)$ is
$$\lbrace x:x=8+35t\vee x=13+35t\vee x=22+35t\vee x=27+35t,t\in\mathbb{Z}\rbrace$$


\end{document}
