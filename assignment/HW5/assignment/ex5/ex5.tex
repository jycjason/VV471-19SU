\documentclass[a4paper,12pt,titlepage]{article}
\usepackage{amsmath} 
\usepackage{amssymb}
\usepackage[nottoc]{tocbibind}
\usepackage{mathrsfs}
\usepackage{float}
\usepackage{indentfirst}
\author{\textit{Jiang Yicheng}\\\textit{515370910224}}
\title{\textbf{VE203\\
		Assignment 5}}
\date{\today}
\usepackage{dsfont}
\usepackage[top=0.8in, bottom=0.8in, left= 2cm, right=2cm]{geometry}
\usepackage{fancyhdr,lastpage}
	\pagestyle{fancy}
	\fancyhf{}
\cfoot{Page \thepage\ of \pageref{LastPage}}
\usepackage{multirow}
\usepackage{gauss}
\usepackage[colorlinks=true,linkcolor=black]{hyperref}
\usepackage[linesnumbered,ruled,longend]{algorithm2e}
\SetKwInOut{Input}{Input}
\SetKwInOut{Output}{Output}
\newcommand{\To}{\KwTo}
\newcommand{\Ret}{\KwRet}
\SetKwProg{Fn}{Function}{\string:}{end}
\SetKwFunction{stgcd}{StGCD}
\newcommand{\udots}{\mathinner{\mskip1mu\raise1pt\vbox{\kern7pt\hbox{.}}\mskip2mu\raise4pt\hbox{.}\mskip2mu\raise7pt\hbox{.}\mskip1mu}}
\SetKwFunction{algohw}{AlgoHw}
\begin{document}

\maketitle

\section*{Exercise 5.1 Binary Insertion Sort} 
\subsection*{i)}

\begin{algorithm}[H]
\Input{$a_1,\cdots , a_n$, $n$ unsorted elements}
\Output{all the $a_i$ , $1 \leqslant i \leqslant n$ in increasing order}
\For{$j\leftarrow$ 2 \To $n$}{
	$i\leftarrow 1$ \;
	$k\leftarrow j$\;
	\While{$i<k$}{
		$m \leftarrow \lfloor (i+k)/2\rfloor$ \;
		\uIf{$a_m>a_j$}{
			$k\leftarrow m$\;
		}
		\Else {$i\leftarrow m+1$\;}
	}
	$l\leftarrow a_j$ \;
	\For{$k\leftarrow$ 0 \To $j-i-1$} {
$a_{j-k}\leftarrow a_{j-k-1}$\;
	}
	$a_i\leftarrow l$\;
}

\Ret{$(a_1,\cdots,a_n)$ in increasing order}

\caption{Binary Insertion Sort}
\end{algorithm}
\subsection*{ii)}
For insertion sort (according to the algorithm in slides), 
\begin{enumerate}
\item $4:4-7\rightarrow 4,7,3,8,1,5,4,2$
\item $3:3-4\rightarrow 3,4,7,8,1,5,4,2$
\item $8:8-3,8-4,8-7,8-8\rightarrow 3,4,7,8,1,5,4,2$
\item $1:1-3\rightarrow 1,3,4,7,8,5,4,2$
\item $5:5-1,5-3,5-4,5-7\rightarrow 1,3,4,5,7,8,4,2$
\item $4:4-1,4-3,4-4\rightarrow 1,3,4,4,5,7,8,2$
\item $2:2-1,2-3\rightarrow 1,2,3,4,4,5,7,8$
\end{enumerate}
So the number of comparisons of elements used by the insertion sort is 16.

For binary insertion sort, 
\begin{enumerate}
\item $4:4-7\rightarrow 4,7,3,8,1,5,4,2$
\item $3:3-7,3-4\rightarrow 3,4,7,8,1,5,4,2$
\item $8:8-4,8-7\rightarrow 3,4,7,8,1,5,4,2$
\item $1:1-7,1-4,1-3\rightarrow 1,3,4,7,8,5,4,2$
\item $5:5-4,5-8,5-7\rightarrow 1,3,4,5,7,8,4,2$
\item $4:4-5,4-3,4-4\rightarrow 1,3,4,4,5,7,8,2$
\item $2:2-4,2-3,2-1\rightarrow 1,2,3,4,4,5,7,8$
\end{enumerate}
So the number of comparisons of elements used by the binary insertion sort is 17.

\subsection*{iii)}
During the $j^{th}(j=2,3,\cdots,n)$ pass, less than or equal to $j$ comparisons are used among these elements. The total number of comparisons $N$ is then
\begin{align*}
N\leqslant\sum\limits_{j=2}^nj=\dfrac{n(n+1)}{2}-1=\dfrac{n^2}{2}+\dfrac{n}{2}-1\leqslant \dfrac{n^2}{2}+\dfrac{n^2}{2}=n^2
\end{align*}
So $N=O(n^2)$ as $n\rightarrow\infty$. So the insertion sort uses $O(n^2)$ comparisons of elements.

\subsection*{iv)}
During the $j^{th}(j=2,3,\cdots,n)$ pass, less than or equal to $1+log_2j$ comparisons are used among these elements. The total number of comparisons $N$ is then
\begin{align*}
N\leqslant\sum\limits_{j=2}^n(1+log_2j)=n-1+log_2n!\leqslant n-1+log_2n^n\leqslant 2nlog_2n
\end{align*}
While for the number of total comparisons $N'$,
$$N'\leqslant\sum\limits_{j=2}^n(1+j-1-1+2log_2j)=\dfrac{n(n-1)}{2}+2log_2n!\leqslant \dfrac{n^2}{2}+2n^2$$ 
So $N'=O(n^2)$ as $n\rightarrow\infty$. So the complexity of binary insertion sort is $O(n^2)$. So it is really faster during the comparison of elements, while it is not significantly faster than the insertion sort in total.

\section*{Exercise 5.2 M,I,C,H,I,G,A,N}
\subsection*{Merge Sort}

\begin{enumerate}
\item $M,I,C,H;I,G,A,N$
\item $M,I;C,H;I,G;A,N$
\item $M-I,C-H,I-G,A-N\rightarrow I,M;C,H;G,I;A,N$
\item $I-C,I-H,G-A,G-N,I-N\rightarrow C,H,I,M;A,G,I,N$
\item $C-A,C-G,H-G,H-I,I-I,I-N,M-N\rightarrow A,C,G,H,I,I,M,N$
\end{enumerate}
So the number of comparisons of elements used by the merge sort is 16.

\subsection*{Insertion Sort}
(according to the algorithm in slides)
\begin{enumerate}
\item $I-M\rightarrow I,M,C,H,I,G,A,N$
\item $C-I\rightarrow C,I,M,H,I,G,A,N$
\item $H-C,H-I\rightarrow C,H,I,M,I,G,A,N$
\item $I-C,I-H,I-I\rightarrow C,H,I,I,M,G,A,N$
\item $G-C,G-H\rightarrow C,G,H,I,I,M,A,N$
\item $A-C\rightarrow A,C,G,H,I,I,M,N$
\item $N-A,N-C,N-G,N-H,N-I,N-I,N-M,N-N\rightarrow A,C,G,H,I,I,M,N$
\end{enumerate}
So the number of comparisons of elements used by the insertion sort is 18.


\subsection*{Bubble Sort}
\begin{enumerate}
\item 
\begin{enumerate}
\item $M-I\rightarrow I,M,C,H,I,G,A,N$
\item $M-C\rightarrow I,C,M,H,I,G,A,N$
\item $M-H\rightarrow I,C,H,M,I,G,A,N$
\item $M-I\rightarrow I,C,H,I,M,G,A,N$
\item $M-G\rightarrow I,C,H,I,G,M,A,N$
\item $M-A\rightarrow I,C,H,I,G,A,M,N$
\item $M-N\rightarrow I,C,H,I,G,A,M,N$
\end{enumerate}
\item 
\begin{enumerate}
\item $I-C\rightarrow C,I,H,I,G,A,M,N$
\item $I-H\rightarrow C,H,I,I,G,A,M,N$
\item $I-I\rightarrow C,H,I,I,G,A,M,N$
\item $I-G\rightarrow C,H,I,G,I,A,M,N$
\item $I-A\rightarrow C,H,I,G,A,I,M,N$
\item $I-M\rightarrow C,H,I,G,A,I,M,N$
\end{enumerate}
\item 
\begin{enumerate}
\item $C-H\rightarrow C,H,I,G,A,I,M,N$
\item $H-I\rightarrow C,H,I,G,A,I,M,N$
\item $I-G\rightarrow C,H,G,I,A,I,M,N$
\item $I-A\rightarrow C,H,G,A,I,I,M,N$
\item $I-I\rightarrow C,H,G,A,I,I,M,N$
\end{enumerate}
\item 
\begin{enumerate}
\item $C-H\rightarrow C,H,G,A,I,I,M,N$
\item $H-G\rightarrow C,G,H,A,I,I,M,N$
\item $H-A\rightarrow C,G,A,H,I,I,M,N$
\item $H-I\rightarrow C,G,A,H,I,I,M,N$
\end{enumerate}
\item 
\begin{enumerate}
\item $C-G\rightarrow C,G,A,H,I,I,M,N$
\item $G-A\rightarrow C,A,G,H,I,I,M,N$
\item $G-H\rightarrow C,A,G,H,I,I,M,N$
\end{enumerate}
\item 
\begin{enumerate}
\item $C-A\rightarrow A,C,G,H,I,I,M,N$
\item $C-G\rightarrow A,C,G,H,I,I,M,N$
\end{enumerate}
\item 
\begin{enumerate}
\item $A-C\rightarrow A,C,G,H,I,I,M,N$
\end{enumerate}
\end{enumerate}
So the number of comparisons of elements used by the bubble sort is 28.

\section*{Exercise 5.3 Application to Feng Shui}
\subsection*{i)}
For $\underbrace{10^{10^{\udots^{10}}}}_{k\times 10}\leqslant n<\underbrace{10^{10^{\udots^{10}}}}_{(k+1)\times 10}$, the number of decimal digits of $n$ is $\lfloor log_{10}n\rfloor+1$. So after one operation, the sum of these decimal digits at most
$$n_1=9\times(\lfloor log_{10}n\rfloor+1)\in [9\cdot(\underbrace{10^{10^{\udots^{10}}}}_{(k-1)\times 10}+1),9\cdot(\underbrace{10^{10^{\udots^{10}}}}_{k\times 10}+1))\subset[\underbrace{10^{10^{\udots^{10}}}}_{(k-1)\times 10},9\cdot(\underbrace{10^{10^{\udots^{10}}}}_{k\times 10}+2))$$
so the number of decimal digits of $n_1$ is $\lfloor log_{10}n_1\rfloor+1$. So after one operation, the sum of these decimal digits at most
\begin{align*}
n_2&=9\times(\lfloor log_{10}n_1\rfloor+1)\in (9\cdot(\underbrace{10^{10^{\udots^{10}}}}_{(k-2)\times 10}+1),9\cdot(\lfloor log_{10}9\cdot(\underbrace{10^{10^{\udots^{10}}}}_{k\times 10}+2)\rfloor)+1)\\
&\subset[\underbrace{10^{10^{\udots^{10}}}}_{(k-2)\times 10},9\cdot(1+\lfloor log_{10}(\underbrace{10^{10^{\udots^{10}}}}_{k\times 10}+2)\rfloor)+1)\\
&\subset[\underbrace{10^{10^{\udots^{10}}}}_{(k-2)\times 10},9\cdot(1+\lfloor log_{10}(\underbrace{10^{10^{\udots^{10}}}}_{k\times 10}+2)\rfloor)+1)
\\
&=[\underbrace{10^{10^{\udots^{10}}}}_{(k-2)\times 10},9\cdot(\underbrace{10^{10^{\udots^{10}}}}_{(k-1)\times 10}+2))
\end{align*}
then with the similar process, we can prove that
$$n_i\in[\underbrace{10^{10^{\udots^{10}}}}_{(k-i)\times 10},9\cdot(\underbrace{10^{10^{\udots^{10}}}}_{(k+1-i)\times 10}+2))$$
so after $k$ times, 
$$n_k\in[1,9\cdot(10+2))=[1,108)$$
then
$$n_{k+1}\in[1,18],n_{k+2}\in[1,9]$$

So the worst-case number of additions that need to be performed to calculate the iterated integer sum
of some $n$ that $\underbrace{10^{10^{\udots^{10}}}}_{k\times 10}\leqslant n<\underbrace{10^{10^{\udots^{10}}}}_{(k+1)\times 10}$ in this way is the $k+2$.

\subsection*{ii)}
The iterated integer sum of a number $n$ is equal to $n\,\,\, mod\,\,\, 9$.
\subsubsection*{Proof:}
We first prove that the sum of decimal digits of $n$ is congruent to $n$ modulo $9$.

$\forall n\in\mathbb{N}$, we can set $n=a_k\cdot 10^k+a_{k-1}\cdot10^{k-1}+\cdots a_1\cdot 10 +a_0,a_0,a_1,\cdots,a_k\in[0,9]\cap\mathbb{N}$. Then
\begin{align*}
n&\equiv a_k\cdot 10^k+a_{k-1}\cdot10^{k-1}+\cdots a_1\cdot 10 +a_0\equiv a_k\cdot 1^k+a_{k-1}\cdot1^{k-1}+\cdots a_1\cdot 1 +a_0\\ 
&\equiv a_k+a_{k-1}+\cdots a_1 +a_0\,\,(mod\,\,\,9)
\end{align*}
So the sum of decimal digits of $n$ ($n_1$) is congruent to $n$ modulo $9$. So
$$n\equiv n_1\equiv n_2\equiv\cdots \equiv the\,\,\, iterated \,\,\,integer\,\,\, sum\,\,\, of\,\,\,n\,\,\,(mod\,\,\,9)$$
Since the iterated integer sum of $n$ is in $[0,9]$, then the sum is just equal to $n\,\,\,mod\,\,\,9$ 


To sum up, the iterated integer sum of a number $n$ is equal to $n\,\,\, mod\,\,\, 9$.

\subsection*{iii)}
The iterated integer sum of a number $n$ represented in arbitrary base $b$ is equal to $n\, mod\, b-1$.
\subsubsection*{Proof:}
We first prove that the sum of digits of $n$ represented in arbitrary base $b$ is congruent to $n$ modulo $b-1$.

$\forall n\in\mathbb{N}$, we can set $n=a_k\cdot b^k+a_{k-1}\cdot b^{k-1}+\cdots a_1\cdot b +a_0,a_0,a_1,\cdots,a_k\in[0,b-1]\cap\mathbb{N}$. Then
\begin{align*}
n&\equiv a_k\cdot b^k+a_{k-1}\cdot b^{k-1}+\cdots a_1\cdot b +a_0\equiv a_k\cdot 1^k+a_{k-1}\cdot1^{k-1}+\cdots a_1\cdot 1 +a_0\\ 
&\equiv a_k+a_{k-1}+\cdots a_1 +a_0\,\,(mod\,\,\,b-1)
\end{align*}
So the sum of digits of $n$ represented in arbitrary base $b$ is congruent to $n$ modulo $b-1$. So
$$n\equiv n_1\equiv n_2\equiv\cdots \equiv the\,\,\, iterated \,\,\,integer\,\,\, sum\,\,\, of\,\,\,n\,\,\,represented\,\,\, in\,\,\, base\,\,\,b(mod\,\,\,b-1)$$
Since the iterated integer sum of $n$ is in $[0,b-1]$, then the sum is just equal to $n\,\,\,mod\,\,\,b-1$ 


To sum up, the iterated integer sum of a number $n$ represented in arbitrary base $b$ is equal to $n\,\,\, mod\,\,\, b-1$.


\section*{Exercise 5.4 Modular Exponentiation}
$1042=1024+16+2=(10000010010)_2,4102\equiv74\,\,\,(mod\,\,\,2014)$. Let $power=74,x=1$.
\begin{enumerate}
\item $i=0,n_i=0$, $power\cdot power \equiv1448 \,\,\,(mod\,\,\,2014)$, let $power=1448$
\item $i=1,n_i=1$, let $x=1\cdot1448=1448$, $power\cdot power \equiv2096704\equiv130 \,\,\,(mod\,\,\,2014)$, let $power=130$
\item $i=2,n_i=0$, $power\cdot power \equiv16900\equiv788 \,\,\,(mod\,\,\,2014)$, let $power=788$
\item $i=3,n_i=0$, $power\cdot power \equiv620944\equiv632 \,\,\,(mod\,\,\,2014)$, let $power=632$
\item $i=4,n_i=1$, let $x=(1448\cdot 632)\,\,\,mod\,\,\,2014=780$, $power\cdot power \equiv399424\equiv652 \,\,\,(mod\,\,\,2014)$, let $power=652$
\item $i=5,n_i=0$, $power\cdot power \equiv425104\equiv150 \,\,\,(mod\,\,\,2014)$, let $power=150$
\item $i=6,n_i=0$, $power\cdot power \equiv22500\equiv346 \,\,\,(mod\,\,\,2014)$, let $power=346$
\item $i=7,n_i=0$, $power\cdot power \equiv119716\equiv890 \,\,\,(mod\,\,\,2014)$, let $power=890$
\item $i=8,n_i=0$, $power\cdot power \equiv792100\equiv598 \,\,\,(mod\,\,\,2014)$, let $power=598$
\item $i=9,n_i=0$, $power\cdot power \equiv357604\equiv1126 \,\,\,(mod\,\,\,2014)$, let $power=1126$
\item $i=10,n_i=1$, let $x=(780\cdot 1126)\,\,\,mod\,\,\,2014=176$
\end{enumerate}
To sum up, $4102^{1042}\,\,\,mod\,\,\,2014=176.$

\section*{Exercise 5.5 Stein's Algorithm for the GCD in base 2}
\subsection*{i)}
For multiplication by 2, it's just add a 0 after the first digit from right of a number in base 2; for division, it's just add a decimal point between the first and second digit from right of a number in base 2 (or move the decimal point to left for a digit). 
\subsection*{ii)}
According to Lemma 1.6.22, if $a$ and $b$ are both even, then$$gcd(a,b)=2\cdot gcd(\dfrac{a}{2},\dfrac{b}{2})$$
\subsection*{iii)}
If $a$ and $b$ are both odd, then
$$gcd(a-b,b)=gcd(a,b)$$
\subsection*{iv)}
\begin{algorithm}[H]
\Input{Two natrual number $a,b$}
\Output{The greatest common divisor of $a,b$}
\Fn{\stgcd{a, b}}{
	\If{$a=b$} {\Ret{a}}
	\If{$a=0\wedge b=0$} {\Ret{0}}
	\If{$a=0$} {\Ret{b}}
	\If{$b=0$} {\Ret{a}}
	\If{$a$ is even and $b$ is even} {\Ret{$2\times$\stgcd{$\dfrac{a}{2}, \dfrac{b}{2}$}}}
	\If{$a$ is even and $b$ is odd} {\Ret{\stgcd{$\dfrac{a}{2}, b$}}}
	\If{$a$ is odd and $b$ is even} {\Ret{\stgcd{$a, \dfrac{b}{2}$}}}
	\If{$a$ is odd and $b$ is odd} {
		\uIf{$a>b$} {\Ret{\stgcd{$\dfrac{a-b}{2}, b$}}}
		\Else {\Ret{\stgcd{$a, \dfrac{b-a}{2}$}}}
	}		
}
\caption{Algorithm to calculate the gcd of two natural numbers in base 2.}
\end{algorithm}
\section*{Exercise 5.6}
\subsection*{i)$a_n=a_{n-1}+6a_{n-2}$}
Solve $r^2-r-6=0$ and we get $r_1=3,r_2=-2$. So set the solution of the recurrence relation is
$$a_n=\alpha_1 \cdot 3^n+\alpha_2\cdot(-2)^n$$
Since $a_0=3,a_1=6$, then
$$\left\{
\begin{aligned}
&\alpha_1+\alpha_2=3\\
&3\alpha_1-2\alpha_2=6\\
\end{aligned}
\right.\Leftrightarrow \left\{
\begin{aligned}
&\alpha_1=\dfrac{12}{5}\\
&\alpha_2=\dfrac{3}{5}\\
\end{aligned}
\right. $$	
So the solution of the recurrence relation $a_n=a_{n-1}+6a_{n-2}$ is
$$a_n=\dfrac{12}{5} \cdot 3^n+\dfrac{3}{5}\cdot(-2)^n$$

\subsection*{ii)$a_{n+2}=-4a_{n+1}+5a_{n}$}
Solve $r^2+4r-5=0$ and we get $r_1=-5,r_2=1$. So set the solution of the recurrence relation is
$$a_n=\alpha_1 \cdot 1^n+\alpha_2\cdot(-5)^n$$
Since $a_0=2,a_1=8$, then
$$\left\{
\begin{aligned}
&\alpha_1+\alpha_2=2\\
&\alpha_1-5\alpha_2=8\\
\end{aligned}
\right.\Leftrightarrow \left\{
\begin{aligned}
&\alpha_1=3\\
&\alpha_2=-1\\
\end{aligned}
\right. $$	
So the solution of the recurrence relation $a_{n+2}=-4a_{n+1}+5a_{n}$ is
$$a_n=3-(-5)^n$$


\section*{Exercise 5.7}

First, we show that $a_n=\alpha_1\cdot r_0^n+\alpha_2\cdot nr_0^n,\alpha_1,\alpha_2\in\mathbb{R},n\in\mathbb{N}$ solves the recurrence relation $a_n=c_1a_{n-1}+c_2a_{n-2}$, where $r_0^2-c_1r_0-c_2=0,c_1^2+4c_2=0$. So $r_0=\dfrac{c_1}{2}$

$\forall n\in\mathbb{N},$
\begin{align*}
&a_{n+2}-c_1a_{n+1}-c_2a_n\\
=&\alpha_1\cdot r_0^n\underbrace{(r_0^2-c_1r_0-c_2)}_0+\alpha_2\cdot r_0^n((n+2)r_0^2-c_1(n+1)r_0-c_2n)\\
=&\alpha_2\cdot r_0^n(n\cdot\underbrace{(r_0^2-c_1r_0-c_2)}_0+(2(c_1/2)^2-c_1\cdot c_1/2))\\
=&0
\end{align*}
so the recurrence relation is satisfied.

Now let $(a_n)$ be a solution to the recurrence relation. By Lemma 2.3.4 this sequence is unique and determined by $a_0$ and $a_1$. We thus need to show that we can find $\alpha_1$ and $\alpha_2$ such that
$$a_0=\alpha_1,a_1=\alpha_1r_0+\alpha_2r_0$$
and we get that 
$$\alpha_1=a_0,\alpha_2=\dfrac{a_1}{r_0}-\alpha_1\,\,\,(r_0\neq0)$$
$$\alpha_1=a_0,\alpha_2\,\,is\,\,no\,\,need\,\,(r_0=0)$$

To sum up, all solution to a linear homogeneous recurrence relation of degree two are of the form
$$a_n=\alpha_1\cdot r_0^n+\alpha_2\cdot nr_0^n,\alpha_1,\alpha_2\in\mathbb{R},n\in\mathbb{N}$$
if there is only a single characteristic root $r_0$.

\section*{Exercise 5.8}
\subsection*{i)$a_n=5a_{n-1}-6a_{n-2}+42\cdot 4^n$}
Since $-336\cdot(5\cdot 4^{n-1}-6\cdot 4^{n-2}-4^n)=-42\cdot 8\cdot4^{n-2}\cdot(5\cdot 4-6-16)=42\cdot 4^n$, 
$$a_n=5a_{n-1}-6a_{n-2}+42\cdot 4^n\Leftrightarrow (a_n-336\cdot 4^n)=5(a_{n-1}-336\cdot 4^{n-1})-6(a_{n-2}-336\cdot 4^{n-2})$$
Set $b_n=a_n-336\cdot 4^n$, then $b_n=5b_{n-1}-6b_{n-2}$. Solve $r^2-5r+6=0$ and we get $r_1=2,r_2=3$. So set the solution of the recurrence relation is
$$b_n=\alpha_1 \cdot 2^n+\alpha_2\cdot3^n$$
then
$$\left\{
\begin{aligned}
&\alpha_1+\alpha_2=b_0\\
&2\alpha_1+3\alpha_2=b_1\\
\end{aligned}
\right.\Leftrightarrow \left\{
\begin{aligned}
&\alpha_1=3b_0-b_1\\
&\alpha_2=b_1-2b_0\\
\end{aligned}
\right. $$	
So 
\begin{align*}
a_n&=b_n+336\cdot 4^n=(3b_0-b_1)\cdot 2^n+(b_1-2b_0)\cdot3^n+336\cdot 4^n\\
&=(3(a_0-336)-(a_1-336\cdot4))\cdot 2^n+((a_1-336\cdot4)-2(a_0-336))\cdot3^n+336\cdot 4^n\\
&=(3a_0-a_1+336)\cdot 2^n+(a_1-2a_0-672)\cdot3^n+336\cdot 4^n
\end{align*}

So the solution of the recurrence relation $a_n=5a_{n-1}-6a_{n-2}+42\cdot 4^n$ is
$$a_n=(3a_0-a_1+336)\cdot 2^n+(a_1-2a_0-672)\cdot3^n+336\cdot 4^n$$

\subsection*{ii)$a_n=-5a_{n-1}-6a_{n-2}+2^n+3n$}
Since 
\begin{align*}
&-5(-\dfrac{1}{5}\cdot 2^{n-1}-\dfrac{1}{4}(n-1)-\dfrac{17}{48})-6(-\dfrac{1}{5}\cdot 2^{n-2}-\dfrac{1}{4}(n-2)-\dfrac{17}{48})-(-\dfrac{1}{5}\cdot 2^{n}-\dfrac{1}{4}n-\dfrac{17}{48})\\
=&2^{n-2}(2+6/5+4/5)+n(5/4+3/2+1/4)+(-5/4+85/48-3+17/8+17/48)\\
=&2^n+3n
\end{align*}
then
\begin{align*}
&a_n=-5a_{n-1}-6a_{n-2}+2^n+3n\\
\Leftrightarrow& (a_n-\dfrac{1}{5}\cdot 2^{n}-\dfrac{1}{4}n-\dfrac{17}{48})=-5(a_{n-1}-\dfrac{1}{5}\cdot 2^{n-1}-\dfrac{1}{4}(n-1)-\dfrac{17}{48})\\&-6(a_{n-2}-\dfrac{1}{5}\cdot 2^{n-2}-\dfrac{1}{4}(n-2)-\dfrac{17}{48})
\end{align*} 
Set $b_n=a_n-\dfrac{1}{5}\cdot 2^{n}-\dfrac{1}{4}n-\dfrac{17}{48}$, then $b_n=-5b_{n-1}-6b_{n-2}$. Solve $r^2+5r+6=0$ and we get $r_1=-2,r_2=-3$. So set the solution of the recurrence relation is
$$b_n=\alpha_1 \cdot (-2)^n+\alpha_2\cdot(-3)^n$$
then
$$\left\{
\begin{aligned}
&\alpha_1+\alpha_2=b_0\\
&-2\alpha_1-3\alpha_2=b_1\\
\end{aligned}
\right.\Leftrightarrow \left\{
\begin{aligned}
&\alpha_1=3b_0+b_1\\
&\alpha_2=-b_1-2b_0\\
\end{aligned}
\right. $$	
So 
\begin{align*}
a_n=&b_n+\dfrac{1}{5}\cdot 2^{n}+\dfrac{1}{4}n+\dfrac{17}{48}\\=&(3b_0+b_1)\cdot (-2)^n-(b_1+2b_0)\cdot(-3)^n+\dfrac{1}{5}\cdot 2^{n}+\dfrac{1}{4}n+\dfrac{17}{48}\\
=&(3(a_0-1/5-17/48)+(a_1-2/5-1/4-17/48))\cdot (-2)^n\\&-((a_1-2/5-1/4-17/48)+2(a_0-1/5-17/48))\cdot(-3)^n+\dfrac{1}{5}\cdot 2^{n}+\dfrac{1}{4}n+\dfrac{17}{48}\\
=&(3a_0+a_1-\dfrac{8}{3})\cdot (-2)^n-(a_1+2a_0-\dfrac{169}{80})\cdot3^n+\dfrac{1}{5}\cdot 2^{n}+\dfrac{1}{4}n+\dfrac{17}{48}
\end{align*}

So the solution of the recurrence relation $a_n=-5a_{n-1}-6a_{n-2}+2^n+3n$ is
$$a_n=(3a_0+a_1-\dfrac{8}{3})\cdot (-2)^n-(a_1+2a_0-\dfrac{169}{80})\cdot3^n+\dfrac{1}{5}\cdot 2^{n}+\dfrac{1}{4}n+\dfrac{17}{48}$$

\subsection*{iii)$a_n=7a_{n-1}-16a_{n-2}+12a_{n-3}+n 4^n$}
Since 
\begin{align*}
&7(-16(n-1)+80)\cdot4^{n-1}-16(-16(n-2)+80)\cdot4^{n-2}+12(-16(n-3)+80)\cdot4^{n-3}\\&-(-16n+80)\cdot4^{n}\\
=&4^{n-3}(n\cdot(-7\cdot16^2+16^2\cdot4-12\cdot16+16\cdot64)+
7\cdot16\cdot96-64\cdot112+12\cdot128-80\cdot64)\\
=&n4^n
\end{align*}
then
\begin{align*}
&a_n=7a_{n-1}-16a_{n-2}+12a_{n-3}+n 4^n\\
\Leftrightarrow& (a_n+(-16n+80)\cdot4^{n})=7(a_{n-1}+(-16(n-1)+80)\cdot4^{n-1})\\&-16(a_{n-2}+(-16(n-2)+80)\cdot4^{n-2})+12(a_{n-3}+(-16(n-3)+80)\cdot4^{n-3})
+\end{align*} 
Set $b_n=a_n+(-16n+80)\cdot4^{n}$, then $b_n=7b_{n-1}-16b_{n-2}+12b_{n-3}$. Solve $r^3-7r^2+16r-12=0$ and we get $r_1=2,r_2=2,r_3=3$. So set the solution of the recurrence relation is
$$b_n=\alpha_1 \cdot 2^n+\alpha_2n\cdot2^n+\alpha_3\cdot3^n$$
then
$$\left\{
\begin{aligned}
&\alpha_1+\alpha_3=b_0\\
&2\alpha_1+2\alpha_2+3\alpha_3=b_1\\
&4\alpha_1+8\alpha_2+9\alpha_3=b_2\\
\end{aligned}
\right.\Leftrightarrow \left\{
\begin{aligned}
&\alpha_1=-3b_0+4b_1-b_2=-3a_0+4a_1-a_2+16\\
&\alpha_2=\dfrac{-6b_0+5b_1-b_2}{2}=\dfrac{-6a_0+5a_1-a_2+32}{2}\\
&\alpha_3=4b_0-4b_1+b_2=4a_0-4a_1+a_2+64\\
\end{aligned}
\right. $$	
So 
\begin{align*}
a_n=&b_n-(-16n+80)\cdot4^{n}=(3b_0-b_1)\cdot 2^n+(b_1-2b_0)\cdot3^n+336\cdot 4^n\\
=&(-3a_0+4a_1-a_2+16)\cdot 2^n+\dfrac{-6a_0+5a_1-a_2+32}{2}n\cdot 2^n+(4a_0-4a_1+a_2+64)\cdot3^n\\&-(-16n+80)\cdot4^{n}
\end{align*}

So the solution of the recurrence relation $a_n=7a_{n-1}-16a_{n-2}+12a_{n-3}+n 4^n$ is
$$a_n=((-3a_0+4a_1-a_2+16)+\dfrac{-6a_0+5a_1-a_2+32}{2}n)\cdot 2^n+(4a_0-4a_1+a_2+64)\cdot3^n-(-16n+80)\cdot4^{n}$$





\end{document}
